\chapter{The Katz Model}

\section{Simple systems, 1969-1972}
\label{sec:KatzSimple}
The purpose of the very first Katz approach was to compute the action cross
section for one-hit, one-target systems (see hit theory chapter). The
cross-section describes the virtual size of the subunit (cell, center) and
allows for connecting the ion fluence (i.e. dose) with the reponse. 

Reduces the description of the ion effect to the change in virtual size. Same
statistics for hit etc., just $\sigma$ dependend on ion beam quality.

\begin{equation}
 S(D)=S(\Phi)=S_0 e^{-\frac{D}{D_{37}}}=S_0 e^{-\alpha D}=S_0 e^{-\sigma \Phi}
 \label{eq:KatzSimplePhoton}
\end{equation}

We assume that
\begin{itemize}
  \item the dose in the target is homogeneous, we neglect any more detailed
  knowledge
\end{itemize}

We can then replace the dose $D$ in the photon response
\ref{eq:KatzSimplePhoton} by the local dose from the radial dose distribution
$d(r)$. Integrated according to its probability from minimum radius 0
\footnote{Using the original Katz RDD it is actually not possible to use a
lower limit 0 as the RDD diverges. Instead some cut-off radius, often 0.01
nm, is used. However, this introduces an addition free parameter} this will
yield the ion action cross section $\sigma_I$

\begin{equation}
\label{eq:KatzSimpleCrossSection}
\sigma_I = 2\pi\int_{0}^{R_{max}}{1-e^{d(r)/D_{37}}rdr}
\end{equation}

and can be used instead of $\sigma$ in Eq. \ref{eq:KatzSimplePhoton}.

Show example computation and maybe sketch?



\section{Multi-hit systems, 1972- ??, Walig\'orski implementation (?)}

If the photon response of a system indicates a more complex structure, i.e. a
multi-hit characteristic, two modes of creating an lethal event are possible.

\begin{enumerate}
  \item A 'direct', lethal hit. This is further supposed to follow the
  simple one-hit statistics and mathematical approach given in section
  \ref{sec:KatzSimple}. It scales with the fluence. As it is visualized as a
  direct transversal of an ion through the target, this is called the
  'ion-mode'.
  \item Multi-targets can also suffer 'indirect', sublethal hits, e.g. from two
  separated tracks (or two central low-LET ones?). In this case, the
  adequate photon response (see section REF HIT THEORY). As this is the
  prominient mode under photon irradiation, it is called the 'gamma-mode'.
\end{enumerate}

Thus, in modification to the approach in \ref{eq:KatzSimpleCrossSection}, a
different multi-hit photon-response with characteristic dose $D_0$ has to be
used 

\begin{equation}
S(D)=S_0 (e^{-\frac{D}{D_{0}}})^m
\end{equation}

to compute the ion action cross section

\begin{equation}
\label{eq:KatzSimpleCrossSection}
\sigma_I = 2\pi\int_{0}^{R_{max}}({1-e^{d(r)/D_{37}})^mrdr}
\end{equation}

This give the inactivation of a subtarget? of size $a_0$ approx. 1 $\mu$ m (how
does that fit to the ion/gamma thing above)

Beanbag model

- $\sigma_i$ has to be scaled (Katz, 1985) by
$\sigma_0\cdot(1-e^{Z_{eff}^2/\kappa\beta^2})$ to get from the cross section of
a single subtarget to that of the target. $\sigma_0$ is the projection of all
subtarget areas.

- $\sigma / \sigma_0$ contribution from ion kill. $\Pi_i=e^{-\sigma\Phi}$
- rest gamma kill $\Pi_\gamma=1-(1-e^\frac{D_\gamma}{D_0})^m$

$F_S = \Pi_i \cdot \Pi_\gamma$


\section*{Document status}
\begin{tabular}{l l}
2011.01.21&Created by S. Greilich
\end{tabular} 