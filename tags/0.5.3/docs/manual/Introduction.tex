% File for libamtrack manual
% Copyright 2006, 2010 Steffen Greilich / the libamtrack team
% This file is part of the libAmTrack project (libamtrack.dkfz.org).

\chapter{Introduction}

\la{} is a library of computational routines for the prediction of solid state
detector response and radiobiological effectiveness in proton and ion beams. In
this field, \la{} focusses on methods that are based on widely used amorphous
track models (ATMs) rather than for example microdosimetry models. Direct
comparisons between ATMs are usually hampered by the lack of knowledge on the
details of the modelling and computational procedures. We believe that this
hinders a wider application of ATMs and the advance towards a better accurracy
of their predictions. The \la{} project has therefore been started with the
intention to provide an open-source, freely available, and comprehensive code
for the community. Written in ANSI C, it is designed to be used independent of
which platform the user is working on. Furthermore, the idea behind organizing
\la{} as a library is that the user can access its functionality from whatever
software tools they are using, e.g. R, S-Plus, Python, LabView etc. The
organization and documentation of the code -- albeit still far from being
perfect -- should help to use \la{} for educational purposes as well.

\section{Purpose of this document}
In its first part, we gathered questions and answers that
appear during working with response models for ion radiation. Starting
with a rather unsorted collection, this could evolve to a more structured
document (and of course a textbook in the end ;).

In the second part the actual technical reference for \la is found.

\section*{Document status}
\begin{tabular}{l l}
2010.05.28&Created by S. Greilich\\
2011.01.21&Rearranged by S. Greilich
\end{tabular}