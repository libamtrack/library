\chapter{Describing the photon response of a system}

The photon response describes the essential behaviour of a system towards
radiation. The distribution of ionzing events and release of secondary electrons
is hereby assumed to be homogeneous throughout the detector ('see of
electrons'). This should hold for photon energies above approx. 100 keV down
to a nanometer scale (TODO Refs., experiments, limits). \footnote{For lower
photon energies efficiency effects were observed in detectors. This is due to
the local saturation at the end of the electron tracks now being more
important. Ref Mayank} As the photon response is a bulk phenomenon the size of
the detector / target does not have any influence (as long as normalized).

The representation of photon response is usually different for biological
systems and detectors but they are equivalent and can be easily transformed from
one into each other (Fig).

\section{Survival and activation}
Biological systems under consideration are thought to consists of a large
number $N$ of subentities that respond to radiation in a defined way. For
example, $K$ individuals cells in an irradiated cell culture will loose their
ability to divide (say they are 'killed'). The number of the non-affected cells
$S = N-K$ or rather their fraction $F_S = \frac{S}{N}$ is then called the
'survival fraction'.

While for biological system mostly $F_S$ is reported, detectors
are considered to similarily be an ensemble of functional units. E.g. a
luminescence center in a TLD might be activated due to radiation or not, in
general out of $N$ centers $A$ might be activated and $S$ are still silent.

The relation between the activated/killed and silent/survival fraction is
simply:

\begin{equation}
   F_A(D) = 1 - F_S(D)
\end{equation}

where $F_A(D)$ is the activated fraction for given dose and $F_S(D)$ the
survival fraction. In absolute response, the limit ($S_{max} = S(D\rightarrow
0)$ for survival, $A_{max} = A(D\rightarrow +\infty)$ for activation) have to be
taken into account.

For detector often: $R(D) = const\cdot A_{max}\cdot F_A(D)$ but irrelevant
renormalization due to detector size etc. Relevant only functional relation
$F_A(D)$.


\section{Lethal/activation events}
For photon radiation the number of lethal/activation events can be easily
derived from the reponse function. 

We make two assumptions:
\begin{itemize}
  \item{There are $N$ subunits that are all identical}
  \item{Radiation creates $n$ events in these subunits. These events are
  randomy placed in a uniform manner (see definition of photon radiation above.}
\end{itemize}
The average number of events per unit $\mu = n/N$ depends on
\begin{itemize}
  \item{Fluence/Dose: double the dose, double the events}
  \item{Cross-section = size of units}
  \item{?}
\end{itemize} 

Due to the random nature not all units will have seen the same number of
events. Rather this number will be follow a Poisson distribution $P$ with mean
$\mu$:
\begin{itemize}
  \item{$N_0 = P(n=0|\mu)$ will not have been from any event}
  \item{$N_1 = P(n=1|\mu)$ will have seen one event}
  \item{$N_2 = P(n=2|\mu)$ will have seen two events}
  \item{\ldots}
\end{itemize}

As we are only interested here in activation/lethal events we can coerce all
units that experience MORE THAN ONE event. This is due to the ultimateness 
that one can only be killed once. Thus, with recalling the Possion
distribution
\begin{equation}
  P(n) =\ldots
\end{equation}

the surviving/silent number is:

\begin{equation}
  N_0 = P(n=0|\mu) = \exp{-n}
\end{equation}




\section*{Document status}
\begin{tabular}{l l}
2011.01.23&Created by S. Greilich
\end{tabular} 