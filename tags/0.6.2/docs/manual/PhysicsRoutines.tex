% File for libamtrack manual
% Copyright 2006, 2010 Steffen Greilich / the libamtrack team
% This file is part of the libAmTrack project (libamtrack.dkfz.org).

\chapter{Physics routines}

\la{} contains a number of auxiliary routines handling the physics of ion beams needed. These routines can be used independently from the efficiency methods. They are implemented in \texttt{AT\_PhysicsRoutines.c} and are described in detail in this.

\section{\texttt{AT\_beta\_from\_E}}

Computes the relativistic speed $\beta=\frac{v}{c}$ from a particle's kinetic energy using
\begin{equation}
\beta = \sqrt{1 - \frac{1}{\frac{E}{1.0079\cdot m_p}}}
\end{equation}
Note that this relation is independent from the particle mass.\\

Single version:\\
\begin{tabular}{l l}
\texttt{E\_MeV\_u} & the kinetic energy per nucleon (double) \\
\end{tabular}\\

Multi version:\\
\begin{tabular}{l l}
\texttt{n} & array size (long integer) \\
\texttt{E\_MeV\_u} & the kinetic energy per nucleon (array of double, size \texttt{n}) \\
\texttt{beta} & array for results (array of double, size \texttt{n}) \\
\end{tabular}


\section{\texttt{AT\_effective\_charge\_from\_beta}}

Computes the effective charge of a travelling HCP as a function of its relativistic speed. Due to charge pick-up slower particle might not be fully stripped. Here, the Barkas (ref.?) equation is used:

\begin{equation}
Z_{eff}=Z\cdot(1-e^{-125\cdot\frac{\beta}{Z^{2/3}}})
\end{equation}

Single version:\\

\begin{tabular}{l l}
\texttt{beta} & relativistic speed (double) \\
\texttt{Z} & charge (long integer) \\
\end{tabular}

Multi version:\\

\begin{tabular}{l l}
\texttt{n} & array size (integer) \\
\texttt{beta} & relativistic speed (array of double, size \texttt{n}) \\
\texttt{Z} & charge (array of long integer, size \texttt{n}) \\
\texttt{effective\_charge} & relativistic speed (array of double, size \texttt{n}) \\
\end{tabular}


\section*{Document status}
\begin{tabular}{l l}
2010.05.28&Created by S. Greilich
\end{tabular}