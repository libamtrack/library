\chapter{Basics of response modeling}

\section{Amorphous track models}
ATMs (in a slightly confusing manner also referred to as 'track structure
models') started from the works of Robert Katz in the late 1950s. He and his
team worked on finding magnetic monopoles in nuclear emulsions exposed to
high-altitute cosmic radiation (TODO Ref.). Soon, \ldots

ATMs rely on two major assumptions:

\begin{itemize}
\item{They disregard the stochastic energy deposition pattern by secondary
electrons around the track of heavy charged particles (protons or
ions, HCPs). Instead, the averaged dose $d$ as a function of
distance $r$ from the trajectory is considered. $d(r)$ is mostly referred as the
'radial dose distribution' although the term 'distribution' is not fully correct
(Fig. \ref{fig:TST}).}
\item{The second important assumption is that -- since photons deposit their
energy eventually by electrons as well -- local radiation effects are supposed
to be the same for photons and HCPs. }
\end{itemize}
Thus, the response to irradiation with particles of type $T$ and energy $E$ can
be predicted from the homogenous bulk photon dose response $S_X(D)$ of the
detector system \cite{Katz_et_al_1972, Waligorski_and_Katz_1980}.

\begin{figure*}
	\centering
		\includegraphics[width=1.0\textwidth]{pictures/TrackStructureDetailAndRDD.png}
	\caption{Amorphization of the detailed track structure.}
	\label{fig:TST}
\end{figure*}


\section{Microdosimetric models}
Rossi, Kellerer - Theory of dual radiation action.
Olko

\section{Hybrid models}
Combining dual radiation approach (dose pattern in the sensitive target, the
nucleus e.g.) with ATM (to derive the pattern). Mathematically, a problem to
combine gamma and local dose. Solution discriminates approaches:
\begin{itemize}
  \item{LEM Arc segments, but problem. So: Scholz, 1992\ldots{} 1997 Single
  track, input parameter \cite{Geiss_et_al_1997, Bassler_et_al_2008}.}
  \item{GSM, iGSM/dGSM}
  \item{SPIFF}
\end{itemize}


\section*{Document status}
\begin{tabular}{l l}
2011.01.21&Created by S. Greilich
\end{tabular} 