% File for libamtrack manual
% Copyright 2006, 2010 Steffen Greilich / the libamtrack team
% This file is part of the libAmTrack project (libamtrack.dkfz.org).

\chapter{Installation of \la{}}

The \la{} project is hosted at \texttt{https://github.com/libamtrack/library} using \texttt{git} version control. Regularly, release versions are published.

\section{Installing GNU Scientific Library (GSL)}

A prerequsit for using \la{} is GSL (\texttt{http://www.gnu.org/software/gsl/}) as \la{} uses many of its numerical functions.

\begin{itemize}
\item{For most Linux distributions you can install GSL from repositories. Make sure that the GSL path is added to the library path. If you want to or have to compile \la{} yourself you will have to install the header files for GSL, too, i.e. the developer version.}
\item{For Windows there is a port available (\texttt{http://gnuwin32.sourceforge.net/packages/gsl.htm}). To work correctly, it is necessary that the binary files (\texttt{libgsl.dll} and \texttt{libgslcblas.dll}) are placed in the same directory as \texttt{libamtrack.dll}.}
\end{itemize}

\section{Installing \la{} binaries (for Windows only)}
The easiest way to get \la{} is to download the latest compiled version.

\section{Installing \la{} from latest released source code}

\section{Installing \la{} working version}

\subsection{Using Eclipse IDK}
!repository at: \texttt{https://github.com/libamtrack/library}
!SVNKit instead of subversion (JavaHL!)
!Proxy for private IPs
!GSL path for windows (GCC linker - libraries (gsl, gslcblas), library path (normally C:/Programs/GnuWin32/bin), GCC compiler - include path (normally C:/Programs/GnuWin32/include))
\subsection{Using gcc only}
git clone https://github.com/libamtrack/library.git

\section*{Document status}
\begin{tabular}{l l}
2010.05.28&Created by S. Greilich
\end{tabular}